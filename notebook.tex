
% Default to the notebook output style

    


% Inherit from the specified cell style.




    
\documentclass[11pt]{article}

    
    
    \usepackage[T1]{fontenc}
    % Nicer default font (+ math font) than Computer Modern for most use cases
    \usepackage{mathpazo}

    % Basic figure setup, for now with no caption control since it's done
    % automatically by Pandoc (which extracts ![](path) syntax from Markdown).
    \usepackage{graphicx}
    % We will generate all images so they have a width \maxwidth. This means
    % that they will get their normal width if they fit onto the page, but
    % are scaled down if they would overflow the margins.
    \makeatletter
    \def\maxwidth{\ifdim\Gin@nat@width>\linewidth\linewidth
    \else\Gin@nat@width\fi}
    \makeatother
    \let\Oldincludegraphics\includegraphics
    % Set max figure width to be 80% of text width, for now hardcoded.
    \renewcommand{\includegraphics}[1]{\Oldincludegraphics[width=.8\maxwidth]{#1}}
    % Ensure that by default, figures have no caption (until we provide a
    % proper Figure object with a Caption API and a way to capture that
    % in the conversion process - todo).
    \usepackage{caption}
    \DeclareCaptionLabelFormat{nolabel}{}
    \captionsetup{labelformat=nolabel}

    \usepackage{adjustbox} % Used to constrain images to a maximum size 
    \usepackage{xcolor} % Allow colors to be defined
    \usepackage{enumerate} % Needed for markdown enumerations to work
    \usepackage{geometry} % Used to adjust the document margins
    \usepackage{amsmath} % Equations
    \usepackage{amssymb} % Equations
    \usepackage{textcomp} % defines textquotesingle
    % Hack from http://tex.stackexchange.com/a/47451/13684:
    \AtBeginDocument{%
        \def\PYZsq{\textquotesingle}% Upright quotes in Pygmentized code
    }
    \usepackage{upquote} % Upright quotes for verbatim code
    \usepackage{eurosym} % defines \euro
    \usepackage[mathletters]{ucs} % Extended unicode (utf-8) support
    \usepackage[utf8x]{inputenc} % Allow utf-8 characters in the tex document
    \usepackage{fancyvrb} % verbatim replacement that allows latex
    \usepackage{grffile} % extends the file name processing of package graphics 
                         % to support a larger range 
    % The hyperref package gives us a pdf with properly built
    % internal navigation ('pdf bookmarks' for the table of contents,
    % internal cross-reference links, web links for URLs, etc.)
    \usepackage{hyperref}
    \usepackage{longtable} % longtable support required by pandoc >1.10
    \usepackage{booktabs}  % table support for pandoc > 1.12.2
    \usepackage[inline]{enumitem} % IRkernel/repr support (it uses the enumerate* environment)
    \usepackage[normalem]{ulem} % ulem is needed to support strikethroughs (\sout)
                                % normalem makes italics be italics, not underlines
    

    
    
    % Colors for the hyperref package
    \definecolor{urlcolor}{rgb}{0,.145,.698}
    \definecolor{linkcolor}{rgb}{.71,0.21,0.01}
    \definecolor{citecolor}{rgb}{.12,.54,.11}

    % ANSI colors
    \definecolor{ansi-black}{HTML}{3E424D}
    \definecolor{ansi-black-intense}{HTML}{282C36}
    \definecolor{ansi-red}{HTML}{E75C58}
    \definecolor{ansi-red-intense}{HTML}{B22B31}
    \definecolor{ansi-green}{HTML}{00A250}
    \definecolor{ansi-green-intense}{HTML}{007427}
    \definecolor{ansi-yellow}{HTML}{DDB62B}
    \definecolor{ansi-yellow-intense}{HTML}{B27D12}
    \definecolor{ansi-blue}{HTML}{208FFB}
    \definecolor{ansi-blue-intense}{HTML}{0065CA}
    \definecolor{ansi-magenta}{HTML}{D160C4}
    \definecolor{ansi-magenta-intense}{HTML}{A03196}
    \definecolor{ansi-cyan}{HTML}{60C6C8}
    \definecolor{ansi-cyan-intense}{HTML}{258F8F}
    \definecolor{ansi-white}{HTML}{C5C1B4}
    \definecolor{ansi-white-intense}{HTML}{A1A6B2}

    % commands and environments needed by pandoc snippets
    % extracted from the output of `pandoc -s`
    \providecommand{\tightlist}{%
      \setlength{\itemsep}{0pt}\setlength{\parskip}{0pt}}
    \DefineVerbatimEnvironment{Highlighting}{Verbatim}{commandchars=\\\{\}}
    % Add ',fontsize=\small' for more characters per line
    \newenvironment{Shaded}{}{}
    \newcommand{\KeywordTok}[1]{\textcolor[rgb]{0.00,0.44,0.13}{\textbf{{#1}}}}
    \newcommand{\DataTypeTok}[1]{\textcolor[rgb]{0.56,0.13,0.00}{{#1}}}
    \newcommand{\DecValTok}[1]{\textcolor[rgb]{0.25,0.63,0.44}{{#1}}}
    \newcommand{\BaseNTok}[1]{\textcolor[rgb]{0.25,0.63,0.44}{{#1}}}
    \newcommand{\FloatTok}[1]{\textcolor[rgb]{0.25,0.63,0.44}{{#1}}}
    \newcommand{\CharTok}[1]{\textcolor[rgb]{0.25,0.44,0.63}{{#1}}}
    \newcommand{\StringTok}[1]{\textcolor[rgb]{0.25,0.44,0.63}{{#1}}}
    \newcommand{\CommentTok}[1]{\textcolor[rgb]{0.38,0.63,0.69}{\textit{{#1}}}}
    \newcommand{\OtherTok}[1]{\textcolor[rgb]{0.00,0.44,0.13}{{#1}}}
    \newcommand{\AlertTok}[1]{\textcolor[rgb]{1.00,0.00,0.00}{\textbf{{#1}}}}
    \newcommand{\FunctionTok}[1]{\textcolor[rgb]{0.02,0.16,0.49}{{#1}}}
    \newcommand{\RegionMarkerTok}[1]{{#1}}
    \newcommand{\ErrorTok}[1]{\textcolor[rgb]{1.00,0.00,0.00}{\textbf{{#1}}}}
    \newcommand{\NormalTok}[1]{{#1}}
    
    % Additional commands for more recent versions of Pandoc
    \newcommand{\ConstantTok}[1]{\textcolor[rgb]{0.53,0.00,0.00}{{#1}}}
    \newcommand{\SpecialCharTok}[1]{\textcolor[rgb]{0.25,0.44,0.63}{{#1}}}
    \newcommand{\VerbatimStringTok}[1]{\textcolor[rgb]{0.25,0.44,0.63}{{#1}}}
    \newcommand{\SpecialStringTok}[1]{\textcolor[rgb]{0.73,0.40,0.53}{{#1}}}
    \newcommand{\ImportTok}[1]{{#1}}
    \newcommand{\DocumentationTok}[1]{\textcolor[rgb]{0.73,0.13,0.13}{\textit{{#1}}}}
    \newcommand{\AnnotationTok}[1]{\textcolor[rgb]{0.38,0.63,0.69}{\textbf{\textit{{#1}}}}}
    \newcommand{\CommentVarTok}[1]{\textcolor[rgb]{0.38,0.63,0.69}{\textbf{\textit{{#1}}}}}
    \newcommand{\VariableTok}[1]{\textcolor[rgb]{0.10,0.09,0.49}{{#1}}}
    \newcommand{\ControlFlowTok}[1]{\textcolor[rgb]{0.00,0.44,0.13}{\textbf{{#1}}}}
    \newcommand{\OperatorTok}[1]{\textcolor[rgb]{0.40,0.40,0.40}{{#1}}}
    \newcommand{\BuiltInTok}[1]{{#1}}
    \newcommand{\ExtensionTok}[1]{{#1}}
    \newcommand{\PreprocessorTok}[1]{\textcolor[rgb]{0.74,0.48,0.00}{{#1}}}
    \newcommand{\AttributeTok}[1]{\textcolor[rgb]{0.49,0.56,0.16}{{#1}}}
    \newcommand{\InformationTok}[1]{\textcolor[rgb]{0.38,0.63,0.69}{\textbf{\textit{{#1}}}}}
    \newcommand{\WarningTok}[1]{\textcolor[rgb]{0.38,0.63,0.69}{\textbf{\textit{{#1}}}}}
    
    
    % Define a nice break command that doesn't care if a line doesn't already
    % exist.
    \def\br{\hspace*{\fill} \\* }
    % Math Jax compatability definitions
    \def\gt{>}
    \def\lt{<}
    % Document parameters
    \title{Math 377 Lesson 13 Solution}
    
    
    

    % Pygments definitions
    
\makeatletter
\def\PY@reset{\let\PY@it=\relax \let\PY@bf=\relax%
    \let\PY@ul=\relax \let\PY@tc=\relax%
    \let\PY@bc=\relax \let\PY@ff=\relax}
\def\PY@tok#1{\csname PY@tok@#1\endcsname}
\def\PY@toks#1+{\ifx\relax#1\empty\else%
    \PY@tok{#1}\expandafter\PY@toks\fi}
\def\PY@do#1{\PY@bc{\PY@tc{\PY@ul{%
    \PY@it{\PY@bf{\PY@ff{#1}}}}}}}
\def\PY#1#2{\PY@reset\PY@toks#1+\relax+\PY@do{#2}}

\expandafter\def\csname PY@tok@w\endcsname{\def\PY@tc##1{\textcolor[rgb]{0.73,0.73,0.73}{##1}}}
\expandafter\def\csname PY@tok@c\endcsname{\let\PY@it=\textit\def\PY@tc##1{\textcolor[rgb]{0.25,0.50,0.50}{##1}}}
\expandafter\def\csname PY@tok@cp\endcsname{\def\PY@tc##1{\textcolor[rgb]{0.74,0.48,0.00}{##1}}}
\expandafter\def\csname PY@tok@k\endcsname{\let\PY@bf=\textbf\def\PY@tc##1{\textcolor[rgb]{0.00,0.50,0.00}{##1}}}
\expandafter\def\csname PY@tok@kp\endcsname{\def\PY@tc##1{\textcolor[rgb]{0.00,0.50,0.00}{##1}}}
\expandafter\def\csname PY@tok@kt\endcsname{\def\PY@tc##1{\textcolor[rgb]{0.69,0.00,0.25}{##1}}}
\expandafter\def\csname PY@tok@o\endcsname{\def\PY@tc##1{\textcolor[rgb]{0.40,0.40,0.40}{##1}}}
\expandafter\def\csname PY@tok@ow\endcsname{\let\PY@bf=\textbf\def\PY@tc##1{\textcolor[rgb]{0.67,0.13,1.00}{##1}}}
\expandafter\def\csname PY@tok@nb\endcsname{\def\PY@tc##1{\textcolor[rgb]{0.00,0.50,0.00}{##1}}}
\expandafter\def\csname PY@tok@nf\endcsname{\def\PY@tc##1{\textcolor[rgb]{0.00,0.00,1.00}{##1}}}
\expandafter\def\csname PY@tok@nc\endcsname{\let\PY@bf=\textbf\def\PY@tc##1{\textcolor[rgb]{0.00,0.00,1.00}{##1}}}
\expandafter\def\csname PY@tok@nn\endcsname{\let\PY@bf=\textbf\def\PY@tc##1{\textcolor[rgb]{0.00,0.00,1.00}{##1}}}
\expandafter\def\csname PY@tok@ne\endcsname{\let\PY@bf=\textbf\def\PY@tc##1{\textcolor[rgb]{0.82,0.25,0.23}{##1}}}
\expandafter\def\csname PY@tok@nv\endcsname{\def\PY@tc##1{\textcolor[rgb]{0.10,0.09,0.49}{##1}}}
\expandafter\def\csname PY@tok@no\endcsname{\def\PY@tc##1{\textcolor[rgb]{0.53,0.00,0.00}{##1}}}
\expandafter\def\csname PY@tok@nl\endcsname{\def\PY@tc##1{\textcolor[rgb]{0.63,0.63,0.00}{##1}}}
\expandafter\def\csname PY@tok@ni\endcsname{\let\PY@bf=\textbf\def\PY@tc##1{\textcolor[rgb]{0.60,0.60,0.60}{##1}}}
\expandafter\def\csname PY@tok@na\endcsname{\def\PY@tc##1{\textcolor[rgb]{0.49,0.56,0.16}{##1}}}
\expandafter\def\csname PY@tok@nt\endcsname{\let\PY@bf=\textbf\def\PY@tc##1{\textcolor[rgb]{0.00,0.50,0.00}{##1}}}
\expandafter\def\csname PY@tok@nd\endcsname{\def\PY@tc##1{\textcolor[rgb]{0.67,0.13,1.00}{##1}}}
\expandafter\def\csname PY@tok@s\endcsname{\def\PY@tc##1{\textcolor[rgb]{0.73,0.13,0.13}{##1}}}
\expandafter\def\csname PY@tok@sd\endcsname{\let\PY@it=\textit\def\PY@tc##1{\textcolor[rgb]{0.73,0.13,0.13}{##1}}}
\expandafter\def\csname PY@tok@si\endcsname{\let\PY@bf=\textbf\def\PY@tc##1{\textcolor[rgb]{0.73,0.40,0.53}{##1}}}
\expandafter\def\csname PY@tok@se\endcsname{\let\PY@bf=\textbf\def\PY@tc##1{\textcolor[rgb]{0.73,0.40,0.13}{##1}}}
\expandafter\def\csname PY@tok@sr\endcsname{\def\PY@tc##1{\textcolor[rgb]{0.73,0.40,0.53}{##1}}}
\expandafter\def\csname PY@tok@ss\endcsname{\def\PY@tc##1{\textcolor[rgb]{0.10,0.09,0.49}{##1}}}
\expandafter\def\csname PY@tok@sx\endcsname{\def\PY@tc##1{\textcolor[rgb]{0.00,0.50,0.00}{##1}}}
\expandafter\def\csname PY@tok@m\endcsname{\def\PY@tc##1{\textcolor[rgb]{0.40,0.40,0.40}{##1}}}
\expandafter\def\csname PY@tok@gh\endcsname{\let\PY@bf=\textbf\def\PY@tc##1{\textcolor[rgb]{0.00,0.00,0.50}{##1}}}
\expandafter\def\csname PY@tok@gu\endcsname{\let\PY@bf=\textbf\def\PY@tc##1{\textcolor[rgb]{0.50,0.00,0.50}{##1}}}
\expandafter\def\csname PY@tok@gd\endcsname{\def\PY@tc##1{\textcolor[rgb]{0.63,0.00,0.00}{##1}}}
\expandafter\def\csname PY@tok@gi\endcsname{\def\PY@tc##1{\textcolor[rgb]{0.00,0.63,0.00}{##1}}}
\expandafter\def\csname PY@tok@gr\endcsname{\def\PY@tc##1{\textcolor[rgb]{1.00,0.00,0.00}{##1}}}
\expandafter\def\csname PY@tok@ge\endcsname{\let\PY@it=\textit}
\expandafter\def\csname PY@tok@gs\endcsname{\let\PY@bf=\textbf}
\expandafter\def\csname PY@tok@gp\endcsname{\let\PY@bf=\textbf\def\PY@tc##1{\textcolor[rgb]{0.00,0.00,0.50}{##1}}}
\expandafter\def\csname PY@tok@go\endcsname{\def\PY@tc##1{\textcolor[rgb]{0.53,0.53,0.53}{##1}}}
\expandafter\def\csname PY@tok@gt\endcsname{\def\PY@tc##1{\textcolor[rgb]{0.00,0.27,0.87}{##1}}}
\expandafter\def\csname PY@tok@err\endcsname{\def\PY@bc##1{\setlength{\fboxsep}{0pt}\fcolorbox[rgb]{1.00,0.00,0.00}{1,1,1}{\strut ##1}}}
\expandafter\def\csname PY@tok@kc\endcsname{\let\PY@bf=\textbf\def\PY@tc##1{\textcolor[rgb]{0.00,0.50,0.00}{##1}}}
\expandafter\def\csname PY@tok@kd\endcsname{\let\PY@bf=\textbf\def\PY@tc##1{\textcolor[rgb]{0.00,0.50,0.00}{##1}}}
\expandafter\def\csname PY@tok@kn\endcsname{\let\PY@bf=\textbf\def\PY@tc##1{\textcolor[rgb]{0.00,0.50,0.00}{##1}}}
\expandafter\def\csname PY@tok@kr\endcsname{\let\PY@bf=\textbf\def\PY@tc##1{\textcolor[rgb]{0.00,0.50,0.00}{##1}}}
\expandafter\def\csname PY@tok@bp\endcsname{\def\PY@tc##1{\textcolor[rgb]{0.00,0.50,0.00}{##1}}}
\expandafter\def\csname PY@tok@fm\endcsname{\def\PY@tc##1{\textcolor[rgb]{0.00,0.00,1.00}{##1}}}
\expandafter\def\csname PY@tok@vc\endcsname{\def\PY@tc##1{\textcolor[rgb]{0.10,0.09,0.49}{##1}}}
\expandafter\def\csname PY@tok@vg\endcsname{\def\PY@tc##1{\textcolor[rgb]{0.10,0.09,0.49}{##1}}}
\expandafter\def\csname PY@tok@vi\endcsname{\def\PY@tc##1{\textcolor[rgb]{0.10,0.09,0.49}{##1}}}
\expandafter\def\csname PY@tok@vm\endcsname{\def\PY@tc##1{\textcolor[rgb]{0.10,0.09,0.49}{##1}}}
\expandafter\def\csname PY@tok@sa\endcsname{\def\PY@tc##1{\textcolor[rgb]{0.73,0.13,0.13}{##1}}}
\expandafter\def\csname PY@tok@sb\endcsname{\def\PY@tc##1{\textcolor[rgb]{0.73,0.13,0.13}{##1}}}
\expandafter\def\csname PY@tok@sc\endcsname{\def\PY@tc##1{\textcolor[rgb]{0.73,0.13,0.13}{##1}}}
\expandafter\def\csname PY@tok@dl\endcsname{\def\PY@tc##1{\textcolor[rgb]{0.73,0.13,0.13}{##1}}}
\expandafter\def\csname PY@tok@s2\endcsname{\def\PY@tc##1{\textcolor[rgb]{0.73,0.13,0.13}{##1}}}
\expandafter\def\csname PY@tok@sh\endcsname{\def\PY@tc##1{\textcolor[rgb]{0.73,0.13,0.13}{##1}}}
\expandafter\def\csname PY@tok@s1\endcsname{\def\PY@tc##1{\textcolor[rgb]{0.73,0.13,0.13}{##1}}}
\expandafter\def\csname PY@tok@mb\endcsname{\def\PY@tc##1{\textcolor[rgb]{0.40,0.40,0.40}{##1}}}
\expandafter\def\csname PY@tok@mf\endcsname{\def\PY@tc##1{\textcolor[rgb]{0.40,0.40,0.40}{##1}}}
\expandafter\def\csname PY@tok@mh\endcsname{\def\PY@tc##1{\textcolor[rgb]{0.40,0.40,0.40}{##1}}}
\expandafter\def\csname PY@tok@mi\endcsname{\def\PY@tc##1{\textcolor[rgb]{0.40,0.40,0.40}{##1}}}
\expandafter\def\csname PY@tok@il\endcsname{\def\PY@tc##1{\textcolor[rgb]{0.40,0.40,0.40}{##1}}}
\expandafter\def\csname PY@tok@mo\endcsname{\def\PY@tc##1{\textcolor[rgb]{0.40,0.40,0.40}{##1}}}
\expandafter\def\csname PY@tok@ch\endcsname{\let\PY@it=\textit\def\PY@tc##1{\textcolor[rgb]{0.25,0.50,0.50}{##1}}}
\expandafter\def\csname PY@tok@cm\endcsname{\let\PY@it=\textit\def\PY@tc##1{\textcolor[rgb]{0.25,0.50,0.50}{##1}}}
\expandafter\def\csname PY@tok@cpf\endcsname{\let\PY@it=\textit\def\PY@tc##1{\textcolor[rgb]{0.25,0.50,0.50}{##1}}}
\expandafter\def\csname PY@tok@c1\endcsname{\let\PY@it=\textit\def\PY@tc##1{\textcolor[rgb]{0.25,0.50,0.50}{##1}}}
\expandafter\def\csname PY@tok@cs\endcsname{\let\PY@it=\textit\def\PY@tc##1{\textcolor[rgb]{0.25,0.50,0.50}{##1}}}

\def\PYZbs{\char`\\}
\def\PYZus{\char`\_}
\def\PYZob{\char`\{}
\def\PYZcb{\char`\}}
\def\PYZca{\char`\^}
\def\PYZam{\char`\&}
\def\PYZlt{\char`\<}
\def\PYZgt{\char`\>}
\def\PYZsh{\char`\#}
\def\PYZpc{\char`\%}
\def\PYZdl{\char`\$}
\def\PYZhy{\char`\-}
\def\PYZsq{\char`\'}
\def\PYZdq{\char`\"}
\def\PYZti{\char`\~}
% for compatibility with earlier versions
\def\PYZat{@}
\def\PYZlb{[}
\def\PYZrb{]}
\makeatother


    % Exact colors from NB
    \definecolor{incolor}{rgb}{0.0, 0.0, 0.5}
    \definecolor{outcolor}{rgb}{0.545, 0.0, 0.0}



    
    % Prevent overflowing lines due to hard-to-break entities
    \sloppy 
    % Setup hyperref package
    \hypersetup{
      breaklinks=true,  % so long urls are correctly broken across lines
      colorlinks=true,
      urlcolor=urlcolor,
      linkcolor=linkcolor,
      citecolor=citecolor,
      }
    % Slightly bigger margins than the latex defaults
    
    \geometry{verbose,tmargin=1in,bmargin=1in,lmargin=1in,rmargin=1in}
    
    

    \begin{document}
    
    
    \maketitle
    
    

    
    \subsection{Probability Distributions}\label{probability-distributions}

\subsubsection{Professors Warner and
Horton}\label{professors-warner-and-horton}

In this notebook, we will introduce some of the more common discrete and
continuous probability distributions. These are commonly used to model
random events.

    \subsubsection{Discrete Distributions}\label{discrete-distributions}

    Remember that a random variable is a mapping from the sample space to a
real number, \(X:S \rightarrow \mathbb{R}\). The \emph{probability mass
function} (pmf) is a function that maps each outcome of the random
variable to a real number between 0 and 1,
\(f: \mathbb{R} \rightarrow [0,1]\), such that for all
\(x \in \mathbb{R}\),

\[f(x)=P(X=x)\]

    Another useful idea is that of the \emph{cumulative distribution
function} (cdf) which is:

\[ F_{X}(x) = P(X \leq x)\]

    The pmf can be written as an explicit function, a table, a plot, or
words. We will focus on the first. Most problems take a wording and
convert it to a table of function.

    \subparagraph{Example}\label{example}

A team is playing a two game series against another team. The
probability of winning the first game is 0.52 and the probability of
winning the second, your star player has a planned absence, is 0.43. Let
\(X\) be the number of games won in the two played.

\begin{enumerate}
\def\labelenumi{\arabic{enumi})}
\tightlist
\item
  Determine the probability mass function as a table.\\
\item
  Find the cumulative distribution function as a table.\\
\item
  Find the probability of winning at least one game.
\end{enumerate}

    \begin{Verbatim}[commandchars=\\\{\}]
{\color{incolor}In [{\color{incolor}1}]:} \PY{c+c1}{\PYZsh{} Run this code to import packages}
        \PY{k+kn}{import} \PY{n+nn}{datascience} \PY{k}{as} \PY{n+nn}{ds}
        \PY{k+kn}{import} \PY{n+nn}{numpy} \PY{k}{as} \PY{n+nn}{np}
        \PY{k+kn}{from} \PY{n+nn}{scipy} \PY{k}{import} \PY{n}{stats}
        \PY{k+kn}{import} \PY{n+nn}{matplotlib}\PY{n+nn}{.}\PY{n+nn}{pyplot} \PY{k}{as} \PY{n+nn}{plt}
        \PY{o}{\PYZpc{}}\PY{k}{matplotlib} inline
\end{Verbatim}


    The random variable \(X\) the number of games won in two played. It can
take the values \({0,1,2}\). The sample space is \({LL,LW,WL,WW}\) which
we will map to the random variable. Thus \(LL \rightarrow 0\),
\(LW \rightarrow 1\), \(WL \rightarrow 1\), and \(WW \rightarrow 2\).
The pmf is:

    \begin{enumerate}
\def\labelenumi{\arabic{enumi})}
\tightlist
\item
  Find the probability mass function.
\end{enumerate}

    \begin{Verbatim}[commandchars=\\\{\}]
{\color{incolor}In [{\color{incolor}2}]:} \PY{n}{w1}\PY{o}{=}\PY{o}{.}\PY{l+m+mi}{52}
        \PY{n}{w2}\PY{o}{=}\PY{o}{.}\PY{l+m+mi}{43}
        \PY{n}{pmf\PYZus{}table}\PY{o}{=}\PY{n}{ds}\PY{o}{.}\PY{n}{Table}\PY{p}{(}\PY{p}{)}\PY{o}{.}\PY{n}{with\PYZus{}columns}\PY{p}{(}\PY{p}{[}\PY{l+s+s2}{\PYZdq{}}\PY{l+s+s2}{Number of Wins (x)}\PY{l+s+s2}{\PYZdq{}}\PY{p}{,}\PY{n}{np}\PY{o}{.}\PY{n}{arange}\PY{p}{(}\PY{l+m+mi}{3}\PY{p}{)}\PY{p}{]}\PY{p}{)}
        \PY{n}{probs}\PY{o}{=} \PY{n}{np}\PY{o}{.}\PY{n}{array}\PY{p}{(}\PY{p}{[}\PY{p}{(}\PY{l+m+mi}{1}\PY{o}{\PYZhy{}}\PY{n}{w1}\PY{p}{)}\PY{o}{*}\PY{p}{(}\PY{l+m+mi}{1}\PY{o}{\PYZhy{}}\PY{n}{w2}\PY{p}{)}\PY{p}{,}\PY{p}{(}\PY{l+m+mi}{1}\PY{o}{\PYZhy{}}\PY{n}{w1}\PY{p}{)}\PY{o}{*}\PY{n}{w2}\PY{o}{+}\PY{n}{w1}\PY{o}{*}\PY{p}{(}\PY{l+m+mi}{1}\PY{o}{\PYZhy{}}\PY{n}{w2}\PY{p}{)}\PY{p}{,}\PY{n}{w1}\PY{o}{*}\PY{n}{w2}\PY{p}{]}\PY{p}{)}
        \PY{n}{pmf\PYZus{}table}\PY{o}{=}\PY{n}{pmf\PYZus{}table}\PY{o}{.}\PY{n}{with\PYZus{}columns}\PY{p}{(}\PY{p}{[}\PY{l+s+s2}{\PYZdq{}}\PY{l+s+s2}{Probability P(X=x)}\PY{l+s+s2}{\PYZdq{}}\PY{p}{,}\PY{n}{probs}\PY{p}{]}\PY{p}{)}
        \PY{n}{pmf\PYZus{}table}
\end{Verbatim}


\begin{Verbatim}[commandchars=\\\{\}]
{\color{outcolor}Out[{\color{outcolor}2}]:} Number of Wins (x) | Probability P(X=x)
        0                  | 0.2736
        1                  | 0.5028
        2                  | 0.2236
\end{Verbatim}
            
    \begin{enumerate}
\def\labelenumi{\arabic{enumi})}
\setcounter{enumi}{1}
\tightlist
\item
  Find the cumulative probability distribution.
\end{enumerate}

    \begin{Verbatim}[commandchars=\\\{\}]
{\color{incolor}In [{\color{incolor}3}]:} \PY{k}{def} \PY{n+nf}{cdf}\PY{p}{(}\PY{n}{table}\PY{p}{)}\PY{p}{:}
            \PY{l+s+sd}{\PYZsq{}\PYZsq{}\PYZsq{}Given a two column table with values in first column and probability mass in the second,}
        \PY{l+s+sd}{    return a table of the cdf}
        \PY{l+s+sd}{    table \PYZhy{} Two columns will realizations of the random variable in first and probability in second\PYZsq{}\PYZsq{}\PYZsq{}}
            \PY{n}{temp}\PY{o}{=}\PY{n}{table}\PY{o}{.}\PY{n}{select}\PY{p}{(}\PY{l+m+mi}{0}\PY{p}{)}\PY{o}{.}\PY{n}{with\PYZus{}columns}\PY{p}{(}\PY{l+s+s2}{\PYZdq{}}\PY{l+s+s2}{Cumulative Probability}\PY{l+s+s2}{\PYZdq{}}\PY{p}{,}\PY{n}{np}\PY{o}{.}\PY{n}{cumsum}\PY{p}{(}\PY{n}{table}\PY{p}{[}\PY{l+m+mi}{1}\PY{p}{]}\PY{p}{)}\PY{p}{)}
            \PY{k}{return} \PY{n}{temp}
\end{Verbatim}


    \begin{Verbatim}[commandchars=\\\{\}]
{\color{incolor}In [{\color{incolor}5}]:} \PY{n}{cdf}\PY{p}{(}\PY{n}{pmf\PYZus{}table}\PY{p}{)}
\end{Verbatim}


\begin{Verbatim}[commandchars=\\\{\}]
{\color{outcolor}Out[{\color{outcolor}5}]:} Number of Wins (x) | Cumulative Probability
        0                  | 0.2736
        1                  | 0.7764
        2                  | 1
\end{Verbatim}
            
    \begin{enumerate}
\def\labelenumi{\arabic{enumi})}
\setcounter{enumi}{2}
\tightlist
\item
  Find the probability of winning at least one game.
\end{enumerate}

    \begin{Verbatim}[commandchars=\\\{\}]
{\color{incolor}In [{\color{incolor}7}]:} \PY{l+m+mi}{1}\PY{o}{\PYZhy{}}\PY{n}{pmf\PYZus{}table}\PY{o}{.}\PY{n}{column}\PY{p}{(}\PY{l+m+mi}{1}\PY{p}{)}\PY{o}{.}\PY{n}{take}\PY{p}{(}\PY{l+m+mi}{0}\PY{p}{)}
\end{Verbatim}


\begin{Verbatim}[commandchars=\\\{\}]
{\color{outcolor}Out[{\color{outcolor}7}]:} 0.7263999999999999
\end{Verbatim}
            
    \paragraph{Moments}\label{moments}

    Random variables can be summarized with expected values. In general, let
\(X\) be a discrete random variable with pmf \(f(x)\). Then
\[E(g(X)) = \sum_{x}g(x)f(x)\]

    The mean is \[E(X) = \mu = \sum_{x}xf(x),\] also called the first moment
about the origin, and variance is
\[E((X- \mu)^{2})  = V(X) = \sigma ^{2} =\sum_{x}(x - \mu)^{2}f(x), \]
known as the second moment about the mean.

    For our problem above, find the mean and variance.

    \begin{Verbatim}[commandchars=\\\{\}]
{\color{incolor}In [{\color{incolor}11}]:} \PY{n}{pmf\PYZus{}table}
\end{Verbatim}


\begin{Verbatim}[commandchars=\\\{\}]
{\color{outcolor}Out[{\color{outcolor}11}]:} Number of Wins (x) | Probability P(X=x)
         0                  | 0.2736
         1                  | 0.5028
         2                  | 0.2236
\end{Verbatim}
            
    \begin{Verbatim}[commandchars=\\\{\}]
{\color{incolor}In [{\color{incolor}8}]:} \PY{p}{(}\PY{n}{pmf\PYZus{}table}\PY{o}{.}\PY{n}{column}\PY{p}{(}\PY{l+m+mi}{0}\PY{p}{)}\PY{o}{*}\PY{n}{pmf\PYZus{}table}\PY{o}{.}\PY{n}{column}\PY{p}{(}\PY{l+m+mi}{1}\PY{p}{)}\PY{p}{)}\PY{o}{.}\PY{n}{sum}\PY{p}{(}\PY{p}{)}
\end{Verbatim}


\begin{Verbatim}[commandchars=\\\{\}]
{\color{outcolor}Out[{\color{outcolor}8}]:} 0.95
\end{Verbatim}
            
    \begin{Verbatim}[commandchars=\\\{\}]
{\color{incolor}In [{\color{incolor}9}]:} \PY{p}{(}\PY{p}{(}\PY{n}{pmf\PYZus{}table}\PY{o}{.}\PY{n}{column}\PY{p}{(}\PY{l+m+mi}{0}\PY{p}{)}\PY{o}{\PYZhy{}}\PY{l+m+mf}{0.95}\PY{p}{)}\PY{o}{*}\PY{o}{*}\PY{l+m+mi}{2}\PY{o}{*}\PY{n}{pmf\PYZus{}table}\PY{o}{.}\PY{n}{column}\PY{p}{(}\PY{l+m+mi}{1}\PY{p}{)}\PY{p}{)}\PY{o}{.}\PY{n}{sum}\PY{p}{(}\PY{p}{)}
\end{Verbatim}


\begin{Verbatim}[commandchars=\\\{\}]
{\color{outcolor}Out[{\color{outcolor}9}]:} 0.49470000000000003
\end{Verbatim}
            
    \subsubsection{Named Discrete
Distributions}\label{named-discrete-distributions}

    Helpfully, random variables often take on familiar "named"
distributions. Such distributions occur often enough that we know the
generic form of the pmf, expected value and variance. In these cases, we
can avoid deriving the pmf like we did in 1) above. Below, we introduce
many common "named" discrete distributions.

    \paragraph{Binomial}\label{binomial}

If each trial of an experiment has only two outcomes, arbitrarily named
success and failure, where the probability of success of each trial is
constant and independent of the prior trials and the number of trials is
fixed in advance, then we have a binomial random variable.\\
The general form of the random variable is \(X\) is the number of
successes in \(n\) trials, where \(p\) is the probability of success in
a single trial.

    The closed form solution to the pmf is
\[f(x;n,p) = {n\choose x}p^{x}(1-p)^{n-x}\]

    Write a function to find the pmf of a binomial and then use it to make a
second cdf function. The following function will help.

    \begin{Verbatim}[commandchars=\\\{\}]
{\color{incolor}In [{\color{incolor}10}]:} \PY{k+kn}{from} \PY{n+nn}{math} \PY{k}{import} \PY{n}{factorial}
         
         \PY{k}{def} \PY{n+nf}{choose}\PY{p}{(}\PY{n}{n}\PY{p}{,} \PY{n}{c}\PY{p}{)}\PY{p}{:}
             \PY{l+s+sd}{\PYZsq{}\PYZsq{}\PYZsq{}Number of ways to choose c items from a list of n items.\PYZsq{}\PYZsq{}\PYZsq{}}
             \PY{k}{return} \PY{n}{factorial}\PY{p}{(}\PY{n}{n}\PY{p}{)} \PY{o}{/}\PY{o}{/} \PY{p}{(}\PY{n}{factorial}\PY{p}{(}\PY{n}{n} \PY{o}{\PYZhy{}} \PY{n}{c}\PY{p}{)} \PY{o}{*} \PY{n}{factorial}\PY{p}{(}\PY{n}{c}\PY{p}{)}\PY{p}{)}
\end{Verbatim}


    \begin{Verbatim}[commandchars=\\\{\}]
{\color{incolor}In [{\color{incolor}11}]:} \PY{k}{def} \PY{n+nf}{my\PYZus{}binom}\PY{p}{(}\PY{n}{x}\PY{p}{,}\PY{n}{n}\PY{p}{,}\PY{n}{p}\PY{p}{)}\PY{p}{:}
             \PY{l+s+sd}{\PYZsq{}\PYZsq{}\PYZsq{}pmf of binomial r.v.; returns the probability of x successes out of n trials }
         \PY{l+s+sd}{    where each trial has probability p of success\PYZsq{}\PYZsq{}\PYZsq{}}
             \PY{k}{return} \PY{n}{choose}\PY{p}{(}\PY{n}{n}\PY{p}{,}\PY{n}{x}\PY{p}{)}\PY{o}{*}\PY{n}{p}\PY{o}{*}\PY{o}{*}\PY{n}{x}\PY{o}{*}\PY{p}{(}\PY{l+m+mi}{1}\PY{o}{\PYZhy{}}\PY{n}{p}\PY{p}{)}\PY{o}{*}\PY{o}{*}\PY{p}{(}\PY{n}{n}\PY{o}{\PYZhy{}}\PY{n}{x}\PY{p}{)}
\end{Verbatim}


    \begin{Verbatim}[commandchars=\\\{\}]
{\color{incolor}In [{\color{incolor}12}]:} \PY{n}{my\PYZus{}binom}\PY{p}{(}\PY{l+m+mi}{1}\PY{p}{,}\PY{l+m+mi}{2}\PY{p}{,}\PY{o}{.}\PY{l+m+mi}{25}\PY{p}{)}
\end{Verbatim}


\begin{Verbatim}[commandchars=\\\{\}]
{\color{outcolor}Out[{\color{outcolor}12}]:} 0.375
\end{Verbatim}
            
    \begin{Verbatim}[commandchars=\\\{\}]
{\color{incolor}In [{\color{incolor}13}]:} \PY{k}{def} \PY{n+nf}{binom\PYZus{}table}\PY{p}{(}\PY{n}{n}\PY{p}{,}\PY{n}{p}\PY{p}{)}\PY{p}{:}
             \PY{l+s+sd}{\PYZsq{}\PYZsq{}\PYZsq{}Create the pmf table for a binomial\PYZsq{}\PYZsq{}\PYZsq{}}
             \PY{n}{pmf\PYZus{}table}\PY{o}{=}\PY{n}{ds}\PY{o}{.}\PY{n}{Table}\PY{p}{(}\PY{p}{)}\PY{o}{.}\PY{n}{with\PYZus{}columns}\PY{p}{(}\PY{l+s+s2}{\PYZdq{}}\PY{l+s+s2}{Number of Successes (x)}\PY{l+s+s2}{\PYZdq{}}\PY{p}{,}\PY{n}{np}\PY{o}{.}\PY{n}{arange}\PY{p}{(}\PY{n}{n}\PY{o}{+}\PY{l+m+mi}{1}\PY{p}{)}\PY{p}{)}
             \PY{n}{pmf\PYZus{}table}\PY{o}{=}\PY{n}{pmf\PYZus{}table}\PY{o}{.}\PY{n}{with\PYZus{}columns}\PY{p}{(}\PY{l+s+s2}{\PYZdq{}}\PY{l+s+s2}{Probability P(X=x)}\PY{l+s+s2}{\PYZdq{}}\PY{p}{,}\PY{n}{pmf\PYZus{}table}\PY{o}{.}\PY{n}{apply}\PY{p}{(}\PY{k}{lambda} \PY{n}{x}\PY{p}{:}\PY{n}{my\PYZus{}binom}\PY{p}{(}\PY{n}{x}\PY{p}{,}\PY{n}{n}\PY{p}{,}\PY{n}{p}\PY{p}{)}\PY{p}{,}\PY{l+m+mi}{0}\PY{p}{)}\PY{p}{)}
             \PY{k}{return} \PY{n}{pmf\PYZus{}table}
\end{Verbatim}


    \begin{Verbatim}[commandchars=\\\{\}]
{\color{incolor}In [{\color{incolor}14}]:} \PY{n}{binom\PYZus{}table}\PY{p}{(}\PY{l+m+mi}{4}\PY{p}{,}\PY{o}{.}\PY{l+m+mi}{25}\PY{p}{)}
\end{Verbatim}


\begin{Verbatim}[commandchars=\\\{\}]
{\color{outcolor}Out[{\color{outcolor}14}]:} Number of Successes (x) | Probability P(X=x)
         0                       | 0.316406
         1                       | 0.421875
         2                       | 0.210938
         3                       | 0.046875
         4                       | 0.00390625
\end{Verbatim}
            
    \begin{Verbatim}[commandchars=\\\{\}]
{\color{incolor}In [{\color{incolor}15}]:} \PY{n}{cdf}\PY{p}{(}\PY{n}{binom\PYZus{}table}\PY{p}{(}\PY{l+m+mi}{4}\PY{p}{,}\PY{o}{.}\PY{l+m+mi}{25}\PY{p}{)}\PY{p}{)}
\end{Verbatim}


\begin{Verbatim}[commandchars=\\\{\}]
{\color{outcolor}Out[{\color{outcolor}15}]:} Number of Successes (x) | Cumulative Probability
         0                       | 0.316406
         1                       | 0.738281
         2                       | 0.949219
         3                       | 0.996094
         4                       | 1
\end{Verbatim}
            
    \subparagraph{Checking results}\label{checking-results}

    The scipy package has common "named" distributions built into its stats
module. See the
\href{https://docs.scipy.org/doc/scipy/reference/stats.html}{link} to
the package.

    \begin{Verbatim}[commandchars=\\\{\}]
{\color{incolor}In [{\color{incolor}16}]:} \PY{n}{stats}\PY{o}{.}\PY{n}{binom}\PY{o}{.}\PY{n}{pmf}\PY{p}{(}\PY{l+m+mi}{2}\PY{p}{,}\PY{l+m+mi}{4}\PY{p}{,}\PY{o}{.}\PY{l+m+mi}{25}\PY{p}{)}
\end{Verbatim}


\begin{Verbatim}[commandchars=\\\{\}]
{\color{outcolor}Out[{\color{outcolor}16}]:} 0.21093750000000006
\end{Verbatim}
            
    \begin{Verbatim}[commandchars=\\\{\}]
{\color{incolor}In [{\color{incolor}17}]:} \PY{n}{stats}\PY{o}{.}\PY{n}{binom}\PY{o}{.}\PY{n}{cdf}\PY{p}{(}\PY{l+m+mi}{2}\PY{p}{,}\PY{l+m+mi}{4}\PY{p}{,}\PY{o}{.}\PY{l+m+mi}{25}\PY{p}{)}
\end{Verbatim}


\begin{Verbatim}[commandchars=\\\{\}]
{\color{outcolor}Out[{\color{outcolor}17}]:} 0.94921875
\end{Verbatim}
            
    The mean and variance of a binomial are \[E(X) = np\] and
\[V(X) = np(1-p)\]

    \subparagraph{Randomization}\label{randomization}

The scipy package also allows you to obtain random realizations from a
random variable with a named distribution.

    \begin{Verbatim}[commandchars=\\\{\}]
{\color{incolor}In [{\color{incolor}18}]:} \PY{n}{rands}\PY{o}{=}\PY{n}{stats}\PY{o}{.}\PY{n}{binom}\PY{o}{.}\PY{n}{rvs}\PY{p}{(}\PY{l+m+mi}{4}\PY{p}{,}\PY{l+m+mf}{0.25}\PY{p}{,}\PY{n}{size}\PY{o}{=}\PY{l+m+mi}{1000}\PY{p}{)}
         \PY{n}{rands}\PY{p}{[}\PY{p}{:}\PY{l+m+mi}{10}\PY{p}{]}
\end{Verbatim}


\begin{Verbatim}[commandchars=\\\{\}]
{\color{outcolor}Out[{\color{outcolor}18}]:} array([2, 0, 0, 2, 1, 1, 1, 0, 0, 0])
\end{Verbatim}
            
    Let's look at the mean and variance of our random sample and compare
with the mean and variance above.

    \begin{Verbatim}[commandchars=\\\{\}]
{\color{incolor}In [{\color{incolor}19}]:} \PY{c+c1}{\PYZsh{}Mean of random sample}
         \PY{n+nb}{print}\PY{p}{(}\PY{n}{rands}\PY{o}{.}\PY{n}{mean}\PY{p}{(}\PY{p}{)}\PY{p}{)}
         \PY{c+c1}{\PYZsh{}Population mean}
         \PY{n+nb}{print}\PY{p}{(}\PY{l+m+mi}{4}\PY{o}{*}\PY{l+m+mf}{0.25}\PY{p}{)}
         \PY{c+c1}{\PYZsh{}Var of random sample}
         \PY{n+nb}{print}\PY{p}{(}\PY{n}{rands}\PY{o}{.}\PY{n}{var}\PY{p}{(}\PY{p}{)}\PY{p}{)}
         \PY{c+c1}{\PYZsh{}Population variance}
         \PY{n+nb}{print}\PY{p}{(}\PY{l+m+mi}{4}\PY{o}{*}\PY{l+m+mf}{0.25}\PY{o}{*}\PY{l+m+mf}{0.75}\PY{p}{)}
\end{Verbatim}


    \begin{Verbatim}[commandchars=\\\{\}]
1.038
1.0
0.7645559999999999
0.75

    \end{Verbatim}

    \subparagraph{Example}\label{example}

Suppose you toss a fair die 12 times. Let \(X\) be the number of rolls
that resulted in 1 or 2.

\begin{enumerate}
\def\labelenumi{\arabic{enumi})}
\item
  Find the pmf of \(X\).
\item
  Find the mean and variance of \(X\).
\item
  Find the probability that \(X\) is in the interval \([3,7]\).
\end{enumerate}

    \(X \sim \textsf{Bin}(12,1/3)\)

\(E(X) = 12*1/3 = 4\)

\(V(X) = 12*1/3*2/3 = 2.667\)

    \begin{Verbatim}[commandchars=\\\{\}]
{\color{incolor}In [{\color{incolor}23}]:} \PY{n}{stats}\PY{o}{.}\PY{n}{binom}\PY{o}{.}\PY{n}{pmf}\PY{p}{(}\PY{n}{np}\PY{o}{.}\PY{n}{arange}\PY{p}{(}\PY{l+m+mi}{3}\PY{p}{,}\PY{l+m+mi}{8}\PY{p}{,}\PY{l+m+mi}{1}\PY{p}{)}\PY{p}{,}\PY{l+m+mi}{12}\PY{p}{,}\PY{l+m+mi}{1}\PY{o}{/}\PY{l+m+mi}{3}\PY{p}{)}\PY{o}{.}\PY{n}{sum}\PY{p}{(}\PY{p}{)}
\end{Verbatim}


\begin{Verbatim}[commandchars=\\\{\}]
{\color{outcolor}Out[{\color{outcolor}23}]:} 0.8001189219499428
\end{Verbatim}
            
    \paragraph{Negative Binomial}\label{negative-binomial}

Similar to binomial except that the failures, thus the trials, are
random and the successes are fixed. Notice that the last trial has to be
a success.\\
The general form of the random variable is \(X\) number of failures
until the rth success.\\
The geometric distribution is a special case where r is 1. Since the
trials are not fixed, \(x = 0, 1, 2, 3, ...\)

    The closed form solution to the pmf is
\[f(x;r,p) = {x+r-1\choose x}p^{r}(1-p)^{x}\]

    The mean and variance of a negative binomial are
\[E(X) = \frac{r}{p}-r\] and \[V(X) = \frac{r(1-p)}{p^2}\]

    \subparagraph{Example}\label{example}

Suppose I am a 60\% free throw shooter (the probability of making a free
throw is 0.6, and all my free throws are independent with identical
chance of success). I would like to make 5 free throws. Let \(X\) be the
number of misses before I make 5 free throws.

\begin{enumerate}
\def\labelenumi{\arabic{enumi})}
\item
  Find the mean and variance of \(X\).
\item
  Find the probability I only need 5 attempts to make 5 free throws.
\item
  Find the probability I need at least 10 attempts to make 5 free
  throws.
\end{enumerate}

    \(X\sim \textsf{NegBin}(5,0.6)\)

\(E(X)=5/0.6 - 5 = 2.333\)

    \begin{Verbatim}[commandchars=\\\{\}]
{\color{incolor}In [{\color{incolor}25}]:} \PY{c+c1}{\PYZsh{}Variance}
         \PY{n}{r}\PY{p}{,} \PY{n}{p} \PY{o}{=} \PY{l+m+mi}{5}\PY{p}{,}\PY{l+m+mf}{0.6}
         \PY{n}{vx}\PY{o}{=}\PY{n}{r}\PY{o}{*}\PY{p}{(}\PY{l+m+mi}{1}\PY{o}{\PYZhy{}}\PY{n}{p}\PY{p}{)}\PY{o}{/}\PY{p}{(}\PY{n}{p}\PY{o}{*}\PY{o}{*}\PY{l+m+mi}{2}\PY{p}{)}
         \PY{n}{vx}
\end{Verbatim}


\begin{Verbatim}[commandchars=\\\{\}]
{\color{outcolor}Out[{\color{outcolor}25}]:} 5.555555555555555
\end{Verbatim}
            
    \paragraph{Poisson}\label{poisson}

The Poisson distribution is common for modeling count or arrival data.
It is common to model arrivals using the Poisson process. In this
process, \(X\), the number of arrivals in a specified amount of time,
follows the Poisson distribution with parameter \(\lambda\), the average
number of arrivals in that specified amount of time. For
\(x = 0,1,2,3,...\), the closed form solution to the pmf is \[
f(x;\lambda)=\frac{e^{-\lambda}\lambda^x}{x!}
\]

    The mean and variance are \[E(X)=V(X)=\lambda\]

    \subparagraph{Example}\label{example}

Suppose fleet vehicles arrive to a maintenance garage at an average rate
of 4 per day. Let's model these arrivals using the Poisson process. Let
\(X\) be the number of vehicles that arrive in a five day period.

\begin{enumerate}
\def\labelenumi{\arabic{enumi})}
\item
  What is the distribution (with parameter) of \(X\)?
\item
  Find the probability that fewer than 10 vehicles arrive in a five day
  period.
\item
  Find the probability that at least 18 vehicles arrive in a five day
  period.
\end{enumerate}

    \(X\sim \textsf{Pois}(\lambda=20)\)

    \begin{Verbatim}[commandchars=\\\{\}]
{\color{incolor}In [{\color{incolor}26}]:} \PY{c+c1}{\PYZsh{} number 2, fewer than 10; P(X\PYZlt{}=9)}
         \PY{n+nb}{print}\PY{p}{(}\PY{n}{stats}\PY{o}{.}\PY{n}{poisson}\PY{o}{.}\PY{n}{cdf}\PY{p}{(}\PY{l+m+mi}{9}\PY{p}{,}\PY{l+m+mi}{20}\PY{p}{)}\PY{p}{)}
         
         \PY{c+c1}{\PYZsh{} number 3, at least 18; P(X\PYZgt{}=18) = 1\PYZhy{}P(X\PYZlt{}=17)}
         \PY{n+nb}{print}\PY{p}{(}\PY{l+m+mi}{1}\PY{o}{\PYZhy{}}\PY{n}{stats}\PY{o}{.}\PY{n}{poisson}\PY{o}{.}\PY{n}{cdf}\PY{p}{(}\PY{l+m+mi}{17}\PY{p}{,}\PY{l+m+mi}{20}\PY{p}{)}\PY{p}{)}
\end{Verbatim}


    \begin{Verbatim}[commandchars=\\\{\}]
0.0049954123083075785
0.7029716020753259

    \end{Verbatim}

    \paragraph{Hypergeometric}\label{hypergeometric}

The hypergeometric distribution is similar to the binomial distribution,
but where outcomes are drawn without replacement. \(X\) is the number of
successes out of \(N\) trials, but these trials are selected out of a
finite population containing \(M\) objects with \(n\) successes. We can
think of this as an urn problem similar to those we discussed in the
Lesson 8 notebook. For \(N-(M-n) \leq x \leq \min(n,N)\), \[
f(x;N,n,M)= \frac{{n\choose k}{{M-n}\choose {N-k}}}{M\choose N}
\]

The mean and variance are \[
E(X)=\frac{nN}{M}
\]

and \[
V(X)=\frac{nN(M-n)(M-N)}{M^2(M-1)}
\]

    \subparagraph{Example}\label{example}

Suppose we have an urn containing 6 blue, 9 red and 8 white balls. We
select 6 at random. Let \(X\) be the number of red balls we select in
our sample of 6.

\begin{enumerate}
\def\labelenumi{\arabic{enumi})}
\item
  Find the expected value of \(X\).
\item
  Find \(P(X\geq 2)\).
\item
  Find the variance of \(X\). Verify your calculation by generating a
  large random sample and computing the variance of that random sample.
\end{enumerate}

    \begin{Verbatim}[commandchars=\\\{\}]
{\color{incolor}In [{\color{incolor}33}]:} \PY{n}{N}\PY{p}{,}\PY{n}{n}\PY{p}{,}\PY{n}{M} \PY{o}{=} \PY{l+m+mi}{6}\PY{p}{,}\PY{l+m+mi}{9}\PY{p}{,}\PY{l+m+mi}{23}
         \PY{c+c1}{\PYZsh{} number 1, E(X)}
         \PY{n+nb}{print}\PY{p}{(}\PY{l+s+s2}{\PYZdq{}}\PY{l+s+s2}{Expected value is }\PY{l+s+s2}{\PYZdq{}}\PY{p}{,}\PY{n}{n}\PY{o}{*}\PY{n}{N}\PY{o}{/}\PY{n}{M}\PY{p}{)}
         
         \PY{c+c1}{\PYZsh{} number 2, P(X\PYZgt{}=2) or 1\PYZhy{}P(X\PYZlt{}=1)}
         \PY{n+nb}{print}\PY{p}{(}\PY{l+s+s2}{\PYZdq{}}\PY{l+s+s2}{P(X\PYZgt{}=2) =}\PY{l+s+s2}{\PYZdq{}}\PY{p}{,}\PY{l+m+mi}{1}\PY{o}{\PYZhy{}}\PY{n}{stats}\PY{o}{.}\PY{n}{hypergeom}\PY{o}{.}\PY{n}{cdf}\PY{p}{(}\PY{l+m+mi}{1}\PY{p}{,}\PY{n}{M}\PY{p}{,}\PY{n}{n}\PY{p}{,}\PY{n}{N}\PY{p}{)}\PY{p}{)}
         
         \PY{c+c1}{\PYZsh{} number 3, variance}
         \PY{n+nb}{print}\PY{p}{(}\PY{l+s+s2}{\PYZdq{}}\PY{l+s+s2}{Variance is }\PY{l+s+s2}{\PYZdq{}}\PY{p}{,} \PY{n}{n}\PY{o}{*}\PY{n}{N}\PY{o}{*}\PY{p}{(}\PY{n}{M}\PY{o}{\PYZhy{}}\PY{n}{n}\PY{p}{)}\PY{o}{*}\PY{p}{(}\PY{n}{M}\PY{o}{\PYZhy{}}\PY{n}{N}\PY{p}{)}\PY{o}{/}\PY{p}{(}\PY{n}{M}\PY{o}{*}\PY{o}{*}\PY{l+m+mi}{2}\PY{o}{*}\PY{p}{(}\PY{n}{M}\PY{o}{\PYZhy{}}\PY{l+m+mi}{1}\PY{p}{)}\PY{p}{)}\PY{p}{)}
         
         \PY{c+c1}{\PYZsh{} verify variance}
         \PY{n+nb}{print}\PY{p}{(}\PY{l+s+s2}{\PYZdq{}}\PY{l+s+s2}{Variance of large sample is }\PY{l+s+s2}{\PYZdq{}}\PY{p}{,} \PY{n}{stats}\PY{o}{.}\PY{n}{hypergeom}\PY{o}{.}\PY{n}{rvs}\PY{p}{(}\PY{n}{M}\PY{p}{,}\PY{n}{n}\PY{p}{,}\PY{n}{N}\PY{p}{,}\PY{n}{size}\PY{o}{=}\PY{l+m+mi}{10000}\PY{p}{)}\PY{o}{.}\PY{n}{var}\PY{p}{(}\PY{p}{)}\PY{p}{)}
\end{Verbatim}


    \begin{Verbatim}[commandchars=\\\{\}]
Expected value is  2.347826086956522
P(X>=2) = 0.7917620137299766
Variance is  1.1043134559202612
Variance of large sample is  1.11567324

    \end{Verbatim}

    \paragraph{Uniform}\label{uniform}

A random variable with the discrete uniform distribution, with
parameters \(a\) and \(b\), has an equal probability of taking on any
integers in the range \([a,b)\). For \(a\leq x < b\), \[
f(x;a,b)= \frac{1}{b-a}
\]

The mean and variance are \[
E(X) = \frac{b+a-1}{2}
\]

and \[
V(X) = \frac{(b-a-1)(b-a+1)}{12}
\]

    \paragraph{Multinomial}\label{multinomial}

The multinomial distribution is a generalization of binomial
distribution. In the binomial case, trials can result in either success
or failure. In the multinomial case, there could be more than two
distinct outcomes. Here, the random variable \(X\) is a vector (rather
than a scalar) representing the counts of each outcome. For more
information regarding the multinomial distribution, please review the
\texttt{scipy} documentation
(https://docs.scipy.org/doc/scipy/reference/generated/scipy.stats.multinomial.html).

    \subsubsection{Continuous Distributions}\label{continuous-distributions}

    Now we turn to random variables that can take on any value in a range
rather than only distinct points. These continous distributions differ
in several ways from their discrete counterparts.

Rather than a pmf, each continuous random variable has a
\textbf{probability density function} (pdf). This too is represented by
\(f_X(x)\), but the function does not return a probability but rather a
density. The pdf has the properties that 1. \(f(x) \geq 0\)

\begin{enumerate}
\def\labelenumi{\arabic{enumi}.}
\setcounter{enumi}{1}
\tightlist
\item
  \(\int_{-\infty}^{\infty} f(y) dy = 1\)
\end{enumerate}

Also, each random variable as a \textbf{cumulative distribution
function} (cdf). This function has the same meaning as in the discrete
case, but can be found via integration:\\
\[ F_{X}(x) = P(X \leq x) = \int_{-\infty}^x f(y)dy \]

    Suppose \(Y\) is a continuous random variable that can take any value in
the interval \([0,1]\). It has the pdf \(f_Y(y) = 2y\) in this interval.
Because \(Y\) is a continuous random variable the probablility at any
specific value is technically 0. We could find the probability that
\(Y\) takes a value in any range by integrating across this function.
For example: \[
P(0\leq Y \leq 0.5) = \int_0^{0.5} 2y dy = y^2\bigg|_0^{0.5} = 0.25
\]

    \begin{Verbatim}[commandchars=\\\{\}]
{\color{incolor}In [{\color{incolor}35}]:} \PY{k+kn}{from} \PY{n+nn}{scipy} \PY{k}{import} \PY{n}{integrate}
\end{Verbatim}


    Create a function for the pdf

    \begin{Verbatim}[commandchars=\\\{\}]
{\color{incolor}In [{\color{incolor}36}]:} \PY{k}{def} \PY{n+nf}{my\PYZus{}pdf}\PY{p}{(}\PY{n}{x}\PY{p}{)}\PY{p}{:}
             \PY{k}{if} \PY{n}{x}\PY{o}{\PYZlt{}}\PY{l+m+mi}{0} \PY{o+ow}{or} \PY{n}{x}\PY{o}{\PYZgt{}}\PY{l+m+mi}{1}\PY{p}{:}
                 \PY{k}{return} \PY{l+m+mi}{0}
             \PY{k}{return} \PY{l+m+mi}{2}\PY{o}{*}\PY{n}{x}
\end{Verbatim}


    \begin{Verbatim}[commandchars=\\\{\}]
{\color{incolor}In [{\color{incolor}39}]:} \PY{n}{np}\PY{o}{.}\PY{n}{round}\PY{p}{(}\PY{n}{integrate}\PY{o}{.}\PY{n}{quad}\PY{p}{(}\PY{k}{lambda} \PY{n}{x}\PY{p}{:}\PY{n}{my\PYZus{}pdf}\PY{p}{(}\PY{n}{x}\PY{p}{)}\PY{p}{,}\PY{l+m+mi}{0}\PY{p}{,}\PY{o}{.}\PY{l+m+mi}{5}\PY{p}{)}\PY{p}{[}\PY{l+m+mi}{0}\PY{p}{]}\PY{p}{,}\PY{l+m+mi}{5}\PY{p}{)}
\end{Verbatim}


\begin{Verbatim}[commandchars=\\\{\}]
{\color{outcolor}Out[{\color{outcolor}39}]:} 0.25
\end{Verbatim}
            
    \begin{Verbatim}[commandchars=\\\{\}]
{\color{incolor}In [{\color{incolor}38}]:} \PY{n}{np}\PY{o}{.}\PY{n}{round}\PY{p}{(}\PY{n}{integrate}\PY{o}{.}\PY{n}{quad}\PY{p}{(}\PY{k}{lambda} \PY{n}{x}\PY{p}{:}\PY{n}{my\PYZus{}pdf}\PY{p}{(}\PY{n}{x}\PY{p}{)}\PY{p}{,}\PY{o}{\PYZhy{}}\PY{l+m+mi}{3}\PY{p}{,}\PY{l+m+mi}{2}\PY{p}{)}\PY{p}{[}\PY{l+m+mi}{0}\PY{p}{]}\PY{p}{,}\PY{l+m+mi}{5}\PY{p}{)}
\end{Verbatim}


\begin{Verbatim}[commandchars=\\\{\}]
{\color{outcolor}Out[{\color{outcolor}38}]:} 1.0
\end{Verbatim}
            
    \subsubsection{Moments}\label{moments}

As in the discrete case, continuous random variables can be summarized
using moments. However, we can use integration rather than direct sums.
Let \(X\) be a continuous random variable with pdf \(f(x)\): \[
E(g(x))=\int_x g(x)f(x)dx
\]

The mean, or expected value, of \(X\) is found by: \[
E(X)=\int_x xf(x)dx
\]

The variance of \(X\) is: \[
V(X)=E((X-\mu)^2)=\int_x (x-\mu)^2f(x) dx
\]

Also, the expression for variance can be simplified to
\(V(X)= E(X^2) - \mu^2\).

    We can use Python to integrate numerically. To find the expected value
of \(X\):

    \begin{Verbatim}[commandchars=\\\{\}]
{\color{incolor}In [{\color{incolor}20}]:} \PY{n}{np}\PY{o}{.}\PY{n}{round}\PY{p}{(}\PY{n}{integrate}\PY{o}{.}\PY{n}{quad}\PY{p}{(}\PY{k}{lambda} \PY{n}{x}\PY{p}{:}\PY{n}{x}\PY{o}{*}\PY{n}{my\PYZus{}pdf}\PY{p}{(}\PY{n}{x}\PY{p}{)}\PY{p}{,}\PY{o}{\PYZhy{}}\PY{l+m+mi}{3}\PY{p}{,}\PY{l+m+mi}{2}\PY{p}{)}\PY{p}{[}\PY{l+m+mi}{0}\PY{p}{]}\PY{p}{,}\PY{l+m+mi}{5}\PY{p}{)}
\end{Verbatim}


\begin{Verbatim}[commandchars=\\\{\}]
{\color{outcolor}Out[{\color{outcolor}20}]:} 0.66667
\end{Verbatim}
            
    \subparagraph{Example}\label{example}

\begin{enumerate}
\def\labelenumi{\arabic{enumi})}
\item
  Let \(g(x)=kx^2\) on \([0,2]\). Find \(k\) that makes \(g(x)\) a
  proper pdf.
\item
  Find \(P(.5<X<1.5)\)
\item
  Find \(E(X)\)
\end{enumerate}

    Need to find \(k\) such that \(\int_0^2 kx^2 dx=1\). \[
\int_0^2 kx^2 dx= \frac{kx^3}{3}\bigg|_0^2 = \frac{8k}{3} = 1
\]

Thus, \(k=\frac{3}{8}\).

    \[
P(0.5<X<1.5) = \int_{0.5}^{1.5} \frac{3x^2}{8} dx = \frac{x^3}{8}\bigg|_{0.5}^{1.5} = \frac{27}{64}-\frac{1}{64} = 0.4063
\]

    \begin{Verbatim}[commandchars=\\\{\}]
{\color{incolor}In [{\color{incolor}42}]:} \PY{k}{def} \PY{n+nf}{my\PYZus{}pdf}\PY{p}{(}\PY{n}{x}\PY{p}{)}\PY{p}{:}
             \PY{k}{if} \PY{n}{x}\PY{o}{\PYZlt{}}\PY{l+m+mi}{0} \PY{o+ow}{or} \PY{n}{x}\PY{o}{\PYZgt{}}\PY{l+m+mi}{2}\PY{p}{:}
                 \PY{k}{return} \PY{l+m+mi}{0}
             \PY{k}{return} \PY{p}{(}\PY{l+m+mi}{3}\PY{o}{/}\PY{l+m+mi}{8}\PY{p}{)}\PY{o}{*}\PY{n}{x}\PY{o}{*}\PY{o}{*}\PY{l+m+mi}{2}
         
         \PY{n}{np}\PY{o}{.}\PY{n}{round}\PY{p}{(}\PY{n}{integrate}\PY{o}{.}\PY{n}{quad}\PY{p}{(}\PY{k}{lambda} \PY{n}{x}\PY{p}{:}\PY{n}{my\PYZus{}pdf}\PY{p}{(}\PY{n}{x}\PY{p}{)}\PY{p}{,}\PY{l+m+mf}{0.5}\PY{p}{,}\PY{l+m+mf}{1.5}\PY{p}{)}\PY{p}{[}\PY{l+m+mi}{0}\PY{p}{]}\PY{p}{,}\PY{l+m+mi}{5}\PY{p}{)}
\end{Verbatim}


\begin{Verbatim}[commandchars=\\\{\}]
{\color{outcolor}Out[{\color{outcolor}42}]:} 0.40625
\end{Verbatim}
            
    \paragraph{Uniform}\label{uniform}

The simplest continuous distribution is the uniform distribution. Let
\(X\) be a random variable that has the continuous uniform distribution
on \([a,b]\). \(X\) has equal probability of taking any value in this
range. For \(a \leq x \leq b\), \[
f(x)=\frac{1}{b-a}
\]

The mean and variance are \[
E(X)=\frac{a+b}{2}
\]

and \[
V(X)=\frac{(b-a)^2}{12}
\]

    \paragraph{Exponential}\label{exponential}

The exponential distribution is closely related to the Poisson
distribution. Both distributions are results of a Poisson process.
Recall that in a Poisson process, if \(X\) represents the number of
arrivals in a certain amount of time, then \(X\) follows a Poisson
distribution with parameter \(\lambda\), the mean number of arrivals in
that amount of time.

Let \(Y\) represent the amount of time until the next arrival in this
process. \(Y\) follows an exponential distribution with parameter
\(\lambda\), the mean number of arrivals in \emph{unit} time. Note that
in \texttt{scipy}, the parameter is specified with the \texttt{scale}
argument, where \texttt{scale} is equal to \(1/\lambda\).

For \(y\geq 0\), \[
f(y)=\lambda e^{-\lambda y}
\]

The mean and variance are \[
E(Y)=\frac{1}{\lambda}
\] and \[
V(Y)=\frac{1}{\lambda^2}
\]

    \subparagraph{Example}\label{example}

Suppose fleet vehicles arrive to a maintenance garage at an average rate
of 4 per day. Let's model these arrivals using the Poisson process. Let
\(Y\) be the time (in days) until the next arrival.

\begin{enumerate}
\def\labelenumi{\arabic{enumi})}
\item
  What is the distribution (with parameter) of \(Y\)?
\item
  Find the probability that no vehicles arrive in the next five days.
\item
  Find the probability that the next vehicle will arrive at least 2 days
  from now, but before 4 days from now.
\end{enumerate}

    \(Y\sim \textsf{Exp}(4)\)

    \begin{Verbatim}[commandchars=\\\{\}]
{\color{incolor}In [{\color{incolor}46}]:} \PY{c+c1}{\PYZsh{}P(Y\PYZgt{}=5) or 1\PYZhy{}P(Y\PYZlt{}5)}
         \PY{n+nb}{print}\PY{p}{(}\PY{l+m+mi}{1}\PY{o}{\PYZhy{}}\PY{n}{stats}\PY{o}{.}\PY{n}{expon}\PY{o}{.}\PY{n}{cdf}\PY{p}{(}\PY{l+m+mi}{5}\PY{p}{,}\PY{n}{scale}\PY{o}{=}\PY{l+m+mi}{1}\PY{o}{/}\PY{l+m+mi}{4}\PY{p}{)}\PY{p}{)}
         
         \PY{c+c1}{\PYZsh{}P(2\PYZlt{}=Y\PYZlt{}4)}
         \PY{n+nb}{print}\PY{p}{(}\PY{n}{stats}\PY{o}{.}\PY{n}{expon}\PY{o}{.}\PY{n}{cdf}\PY{p}{(}\PY{l+m+mi}{4}\PY{p}{,}\PY{n}{scale}\PY{o}{=}\PY{l+m+mi}{1}\PY{o}{/}\PY{l+m+mi}{4}\PY{p}{)}\PY{o}{\PYZhy{}}\PY{n}{stats}\PY{o}{.}\PY{n}{expon}\PY{o}{.}\PY{n}{cdf}\PY{p}{(}\PY{l+m+mi}{2}\PY{p}{,}\PY{n}{scale}\PY{o}{=}\PY{l+m+mi}{1}\PY{o}{/}\PY{l+m+mi}{4}\PY{p}{)}\PY{p}{)}
\end{Verbatim}


    \begin{Verbatim}[commandchars=\\\{\}]
2.0611535811454473e-09
0.00033535009272778193

    \end{Verbatim}

    \paragraph{Normal}\label{normal}

The normal (or Gaussian) distribution is an incredibly important
distribution in probability and statistics. You have certainly seen a
bell curve at some point and are likely aware that certain quantities
take a bell shaped, or normal, distribution.

Let \(X \sim \textsf{N}(\mu,\sigma)\). The parameters \(\mu\) and
\(\sigma\) represent the mean and standard deviation of \(X\). In
\texttt{scipy}, these are referred to as \texttt{loc} and
\texttt{scale}.

For \(-\infty < x < \infty\), \[
f(x;\mu,\sigma) = \frac{1}{\sqrt{2\pi\sigma^2}}e^{-\frac{(x-\mu)^2}{2\sigma^2}}
\]

As mentioned, the mean and variance are \(\mu\) and \(\sigma^2\).

    \subparagraph{Example}\label{example}

Let \(X\) follow the normal distribution with mean 3 and standard
deviation 10.

\begin{enumerate}
\def\labelenumi{\arabic{enumi})}
\item
  Find \(P(X<0)\).
\item
  Find the 90th percentile of \(X\).
\item
  Given \(X\) is greater than the mean, find the probability that \(X\)
  is less than the mean plus two standard deviations.
\end{enumerate}

    \begin{Verbatim}[commandchars=\\\{\}]
{\color{incolor}In [{\color{incolor}50}]:} \PY{c+c1}{\PYZsh{} P(X\PYZlt{}0)}
         \PY{n+nb}{print}\PY{p}{(}\PY{n}{stats}\PY{o}{.}\PY{n}{norm}\PY{o}{.}\PY{n}{cdf}\PY{p}{(}\PY{l+m+mi}{0}\PY{p}{,}\PY{l+m+mi}{3}\PY{p}{,}\PY{l+m+mi}{10}\PY{p}{)}\PY{p}{)}
         
         \PY{c+c1}{\PYZsh{} 90th percentile, or x such that P(X\PYZlt{}x)=0.90}
         \PY{n+nb}{print}\PY{p}{(}\PY{n}{stats}\PY{o}{.}\PY{n}{norm}\PY{o}{.}\PY{n}{ppf}\PY{p}{(}\PY{l+m+mf}{0.9}\PY{p}{,}\PY{l+m+mi}{3}\PY{p}{,}\PY{l+m+mi}{10}\PY{p}{)}\PY{p}{)}
\end{Verbatim}


    \begin{Verbatim}[commandchars=\\\{\}]
0.3820885778110474
15.815515655446003

    \end{Verbatim}

    For number 3, we need to find the conditional probablity: \[
P(X<10+6 | X>3) = \frac{P(3<X<16)}{P(X>3)}
\]

Note that since the normal distribution is symmetric about the mean, we
know that \(P(X>3)=0.5\).

    \begin{Verbatim}[commandchars=\\\{\}]
{\color{incolor}In [{\color{incolor}51}]:} \PY{n+nb}{print}\PY{p}{(}\PY{p}{(}\PY{n}{stats}\PY{o}{.}\PY{n}{norm}\PY{o}{.}\PY{n}{cdf}\PY{p}{(}\PY{l+m+mi}{16}\PY{p}{,}\PY{l+m+mi}{3}\PY{p}{,}\PY{l+m+mi}{10}\PY{p}{)}\PY{o}{\PYZhy{}}\PY{n}{stats}\PY{o}{.}\PY{n}{norm}\PY{o}{.}\PY{n}{cdf}\PY{p}{(}\PY{l+m+mi}{3}\PY{p}{,}\PY{l+m+mi}{3}\PY{p}{,}\PY{l+m+mi}{10}\PY{p}{)}\PY{p}{)}\PY{o}{/}\PY{l+m+mf}{0.5}\PY{p}{)}
\end{Verbatim}


    \begin{Verbatim}[commandchars=\\\{\}]
0.8063990308287794

    \end{Verbatim}

    \paragraph{Gamma}\label{gamma}

    The gamma distribution is a generalization of the exponential
distribution. If \(X\) has the gamma distribution with parameters
\(\alpha\) and \(\lambda\). Note that in \texttt{scipy}, \(\alpha\) is
the shape parameter, \(a\), and \(\lambda\) is the inverse of the scale
parameter (similar to the exponential distribution).

For \(x\geq 0\), \[
f(x;\alpha,\lambda)= \frac{\lambda^\alpha}{\Gamma(\alpha)}x^{\alpha-1}e^{-\lambda x}
\]

The mean and variance are: \[
E(X)=\frac{\alpha}{\lambda}
\]

and \[
V(X)=\frac{\alpha}{\lambda^2}
\]

    \subparagraph{Example}\label{example}

Let \(X \sim \textsf{Gamma}(3,0.7)\).

\begin{enumerate}
\def\labelenumi{\arabic{enumi})}
\item
  Find \(P(X>E(X))\)
\item
  Find \(P(X\leq 1)\)
\end{enumerate}

    \begin{Verbatim}[commandchars=\\\{\}]
{\color{incolor}In [{\color{incolor}53}]:} \PY{c+c1}{\PYZsh{}P(X\PYZgt{}E(X))}
         \PY{n}{a}\PY{p}{,}\PY{n}{l}\PY{o}{=}\PY{l+m+mi}{3}\PY{p}{,}\PY{l+m+mf}{0.7}
         \PY{n}{ex}\PY{o}{=}\PY{n}{a}\PY{o}{/}\PY{n}{l}
         \PY{n+nb}{print}\PY{p}{(}\PY{l+m+mi}{1}\PY{o}{\PYZhy{}}\PY{n}{stats}\PY{o}{.}\PY{n}{gamma}\PY{o}{.}\PY{n}{cdf}\PY{p}{(}\PY{n}{ex}\PY{p}{,}\PY{n}{a}\PY{p}{,}\PY{n}{scale}\PY{o}{=}\PY{l+m+mi}{1}\PY{o}{/}\PY{n}{l}\PY{p}{)}\PY{p}{)}
         
         \PY{c+c1}{\PYZsh{}P(X\PYZlt{}=1)}
         \PY{n+nb}{print}\PY{p}{(}\PY{n}{stats}\PY{o}{.}\PY{n}{gamma}\PY{o}{.}\PY{n}{cdf}\PY{p}{(}\PY{l+m+mi}{1}\PY{p}{,}\PY{n}{a}\PY{p}{,}\PY{n}{scale}\PY{o}{=}\PY{l+m+mi}{1}\PY{o}{/}\PY{n}{l}\PY{p}{)}\PY{p}{)}
\end{Verbatim}


    \begin{Verbatim}[commandchars=\\\{\}]
0.4231900811268434
0.0341415841257085

    \end{Verbatim}

    \begin{Verbatim}[commandchars=\\\{\}]
{\color{incolor}In [{\color{incolor}54}]:} \PY{n}{\PYZus{}}\PY{p}{,}\PY{n}{ax}\PY{o}{=}\PY{n}{plt}\PY{o}{.}\PY{n}{subplots}\PY{p}{(}\PY{l+m+mi}{1}\PY{p}{,}\PY{l+m+mi}{1}\PY{p}{)}
         \PY{n}{x}\PY{o}{=}\PY{n}{np}\PY{o}{.}\PY{n}{arange}\PY{p}{(}\PY{l+m+mf}{0.001}\PY{p}{,}\PY{l+m+mi}{20}\PY{p}{,}\PY{l+m+mf}{0.001}\PY{p}{)}
         \PY{n}{ax}\PY{o}{.}\PY{n}{plot}\PY{p}{(}\PY{n}{x}\PY{p}{,}\PY{n}{stats}\PY{o}{.}\PY{n}{gamma}\PY{o}{.}\PY{n}{pdf}\PY{p}{(}\PY{n}{x}\PY{p}{,}\PY{n}{a}\PY{p}{,}\PY{n}{scale}\PY{o}{=}\PY{l+m+mi}{1}\PY{o}{/}\PY{n}{l}\PY{p}{)}\PY{p}{)}\PY{p}{;}
\end{Verbatim}


    \begin{center}
    \adjustimage{max size={0.9\linewidth}{0.9\paperheight}}{output_90_0.png}
    \end{center}
    { \hspace*{\fill} \\}
    
    Note that we can plot the exponential distribution with \(\lambda=0.7\)
by setting \(a=1\):

    \begin{Verbatim}[commandchars=\\\{\}]
{\color{incolor}In [{\color{incolor}55}]:} \PY{n}{\PYZus{}}\PY{p}{,}\PY{n}{ax}\PY{o}{=}\PY{n}{plt}\PY{o}{.}\PY{n}{subplots}\PY{p}{(}\PY{l+m+mi}{1}\PY{p}{,}\PY{l+m+mi}{1}\PY{p}{)}
         \PY{n}{x}\PY{o}{=}\PY{n}{np}\PY{o}{.}\PY{n}{arange}\PY{p}{(}\PY{l+m+mf}{0.001}\PY{p}{,}\PY{l+m+mi}{10}\PY{p}{,}\PY{l+m+mf}{0.001}\PY{p}{)}
         \PY{n}{ax}\PY{o}{.}\PY{n}{plot}\PY{p}{(}\PY{n}{x}\PY{p}{,}\PY{n}{stats}\PY{o}{.}\PY{n}{gamma}\PY{o}{.}\PY{n}{pdf}\PY{p}{(}\PY{n}{x}\PY{p}{,}\PY{l+m+mi}{1}\PY{p}{,}\PY{n}{scale}\PY{o}{=}\PY{l+m+mi}{1}\PY{o}{/}\PY{n}{l}\PY{p}{)}\PY{p}{)}\PY{p}{;}
\end{Verbatim}


    \begin{center}
    \adjustimage{max size={0.9\linewidth}{0.9\paperheight}}{output_92_0.png}
    \end{center}
    { \hspace*{\fill} \\}
    
    \paragraph{Beta}\label{beta}

The beta distribution is unique among continuous distributions in that
it applies only to random variables restricted to the domain \([0,1]\).
The continuous uniform distribution is a special case of the beta
distribution. To learn more about the beta distribution, see the
\texttt{scipy} documentation:
https://docs.scipy.org/doc/scipy/reference/generated/scipy.stats.beta.html\#scipy.stats.beta

    \subparagraph{Example}\label{example}

\begin{enumerate}
\def\labelenumi{\arabic{enumi})}
\tightlist
\item
  What are the parameters of the beta distribution that make the
  distribution equivalent to \(\textsf{Unif}(0,1)\)?
\end{enumerate}

    When \(a\) and \(b\) are both 1, the pdf of \(X\) reduces to a constant,
which must be 1 in order to result in a valid pdf.


    % Add a bibliography block to the postdoc
    
    
    
    \end{document}
